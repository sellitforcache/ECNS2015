\documentclass[a4paper]{jpconf}
\usepackage{graphicx}
\begin{document}
\title{Upgrades to the SINQ Cold Neutron Source}

\author{R M Bergmann, U Filges, D Kiselev, T Reiss, V Talanov and M Wohlmuther}






\address{Paul Scherrer Institut, 5232 Villigen, Switzerland}

\ead{ryan.bergmann@psi.ch}

\begin{abstract}Changing the configuration of the cold neutron source during an extended shutdown of Swiss Spallation Neutron Source (SINQ) is being considered for improving performance of the instruments that use the cold source.  The cold neutron source consists of a 20 L volume of liquid D2 at approximately 25 K.  Previous upgrades included adding a re-entrant hole into one side of the D2 volume to allow cold neutrons to stream uninhibited from the center of the source towards the instrument neutron guides.  Calculations done prior to these changes predicted cold neutron fluence gains from 1.2 to 1.6, with gains increasing with wavelength. These increases have not been observed, and it is suspected that the re-entrant hole is not fully voided.  Voiding the re-entrant hole relies on radiative heating to boil D2 which in turn fills the re-entrant hole cavity, pushing the liquid D2 out.  Proposed plans include making the re-entrant hole external (ensuring that it is not filled with liquid D2), removing some extra structural material around the D2 tank, redesigning the re-entrant hole geometry to be more optically ideal, introducing a Pb-208 reflector to minimize cold neutron up-scattering from the surrounding D2O moderator tank, and implementing a small liquid H2 volume to provide a cold neutron “hot spot” for certain instruments.  These changes are predicted to increase neutron fluence between 1.1 to 2.0 times the current levels, depending on instrument location, view, and wavelengths of interest.
\end{abstract}

\ack{This work was supported by Swiss National Science Foundation grant 200021\_150048/1.}

\section{Introduction}

\section{Preparing your paper}


\section{The title, authors, addresses and abstract} 


\subsection{Sample coding for the start of an article}

\section{The text}

\section{References}


\subsection{Using \BibTeX}
We highly recommend the {\ttfamily\textbf\selectfont iopart-num} \BibTeX\ package by Mark~A~Caprio \cite{iopartnum}, which is included with this documentation.


\begin{figure}[h]
\begin{minipage}{14pc}
\includegraphics[width=14pc]{name.eps}
\caption{\label{label}Figure caption for first of two sided figures.}
\end{minipage}\hspace{2pc}%
\begin{minipage}{14pc}
\includegraphics[width=14pc]{name.eps}
\caption{\label{label}Figure caption for second of two sided figures.}
\end{minipage} 
\end{figure}

\begin{figure}[h]
\includegraphics[width=14pc]{name.eps}\hspace{2pc}%
\begin{minipage}[b]{14pc}\caption{\label{label}Figure caption for a narrow figure where the caption is put at the side of the figure.}
\end{minipage}
\end{figure}

Using the graphicx package figures can be included using code such as:
\begin{verbatim}
\begin{figure}
\begin{center}
\includegraphics{file.eps}
\end{center}
\caption{\label{label}Figure caption}
\end{figure}
\end{verbatim}

\section*{References}
\begin{thebibliography}{9}
\bibitem{iopartnum} IOP Publishing is to grateful Mark A Caprio, Center for Theoretical Physics, Yale University, for permission to include the {\tt iopart-num} \BibTeX package (version 2.0, December 21, 2006) with  this documentation. Updates and new releases of {\tt iopart-num} can be found on \verb"www.ctan.org" (CTAN). 
\end{thebibliography}

\end{document}


